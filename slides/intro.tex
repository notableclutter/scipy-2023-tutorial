\documentclass[xcolor=svgnames]{beamer}
\usetheme{Torino}

\usepackage{epsfig} %for figures
\usepackage{xcolor} %for color
\usepackage[utf8]{inputenc}
\usepackage{multicol}
\usepackage{hyperref}

% latex definitions:
\def\d{{\rm d}}
\def\half{{\textstyle{1\over2}}}



\title[SymPy\hspace{4em}\insertframenumber/
\inserttotalframenumber]{~\\ SymPy Tutorial \\~}


\author[A. Meurer, Matthew Rocklin, Jason Moore]
{Aaron Meurer, Ondřej Čertík, Amit Kumar, Jason Moore, \\ Sartaj Singh, Harsh Gupta}

\pgfdeclareimage[height=1.5cm]{mylogo}{sympy-250px}
\institute{\pgfuseimage{mylogo}}

\date{July 11, 2016}

\begin{document}

\begin{frame}
  \maketitle
\begin{center}
\normalsize All materials for today's tutorial are at \url{http://www.sympy.org/scipy-2016-tutorial/}
\end{center}
\end{frame}

\begin{frame}{Outline}
  \begin{block}{SymPy Introduction}
    \begin{itemize}
    \item Goal
    \item Features
    \item History
    \item Present
    \item Future
    \end{itemize}
  \end{block}

  \begin{block}{Tutorial}
    \begin{itemize}
    \item Intro to SymPy and Basic features
    \item Solving real life problems
    \end{itemize}
  \end{block}
\end{frame}

\begin{frame}{SymPy Goal}
  \begin{block}{Goal}
    Provide a symbolic manipulation library in Python.
  \end{block}
  \pause
  \begin{block}

    ``SymPy is an open source Python library for symbolic mathematics. It aims to
    become a full-featured computer algebra system (CAS) while keeping the code as
    simple as possible in order to be comprehensible and easily extensible. SymPy
    is written entirely in Python and does not require any external libraries.''

  \end{block}
\end{frame}

\begin{frame}{Why SymPy?}
  \begin{block}{}
    \begin{itemize}
      \item Standalone
      \item Full featured
      \item BSD licensed
      \item Embraces Python
      \item Usable as a library
    \end{itemize}
  \end{block}
\end{frame}

\begin{frame}{Features}
  \begin{multicols}{2}
    \tiny
    \begin{itemize}
    \item \textbf{Core Capabilities}
      \begin{itemize}
        \tiny
      \item Basic arithmetic: Support for operators such as +, -, *, /, ** (power)
      \item Simplification
      \item Expansion
      \item Functions: trigonometric, hyperbolic, exponential, roots, logarithms,
        absolute value, spherical harmonics, factorials and gamma functions, zeta
        functions, polynomials, special functions, \ldots
      \item Substitution
      \item Numbers: arbitrary precision integers, rationals, and floats
      \item Noncommutative symbols
      \item Pattern matching
      \end{itemize}
    \item \textbf{Polynomials}
      \begin{itemize}
        \tiny
      \item Basic arithmetic: division, gcd, \ldots
      \item Factorization
      \item Square-free decomposition
      \item Gröbner bases
      \item Partial fraction decomposition
      \item Resultants
      \end{itemize}
    \item \textbf{Calculus}
      \begin{itemize}
        \tiny
      \item Limits: $\lim_{x\to 0}{x\log(x)} = 0$
      \item Differentiation
      \item Integration: It uses extended Risch-Norman heuristic
      \item Taylor (Laurent) series
      \end{itemize}
    \item \textbf{Solving equations}
      \begin{itemize}
        \tiny
      \item Polynomial equations
      \item Algebraic equations
      \item Differential equations
      \item Difference equations
      \item Systems of equations
      \end{itemize}
    \item \textbf{Combinatorics}
      \begin{itemize}
        \tiny
      \item Permutations
      \item Combinations
      \item Partitions
      \item Subsets
      \item Permutation Groups: Polyhedral, Rubik, Symmetric, \ldots
      \item Prufer and Gray Codes
      \end{itemize}

    \end{itemize}
  \end{multicols}
\end{frame}

\begin{frame}{Features}
  \begin{multicols}{2}
    \begin{itemize}
      \tiny
    \item \textbf{Discrete math}
      \begin{itemize}
        \tiny
      \item Binomial coefficients
      \item Summations
      \item Products
      \item Number theory: generating prime numbers, primality testing, integer
        factorization, \ldots
      \item Logic expressions
      \end{itemize}

    \item \textbf{Matrices}
      \begin{itemize}
        \tiny
      \item Basic arithmetic
      \item Eigenvalues/eigenvectors
      \item Determinants
      \item Inversion
      \item Solving
      \item Abstract expressions
      \end{itemize}


    \item \textbf{Geometric Algebra}


    \item \textbf{Geometry}
      \begin{itemize}
        \tiny
      \item points, lines, rays, segments, ellipses, circles, polygons, \ldots
      \item Intersection
      \item Tangency
      \item Similarity
      \end{itemize}

    \item \textbf{Plotting}
      \begin{itemize}
        \tiny
      \item Coordinate modes
      \item Plotting Geometric Entities
      \item 2D and 3D
      \item Interactive interface
      \item Colors
      \end{itemize}

    \item \textbf{Physics}
      \begin{itemize}
        \tiny
      \item Units
      \item Mechanics
      \item Quantum
      \item Gaussian Optics
      \item Pauli Algebra
      \end{itemize}

    \item \textbf{Statistics}
      \begin{itemize}
        \tiny
      \item Normal distributions
      \item Uniform distributions
      \item Probability
      \end{itemize}

    \item \textbf{Printing}
      \begin{itemize}
        \tiny
      \item Pretty printing: ASCII/Unicode pretty printing, LaTeX
      \item Code generation: C, Fortran, Python
      \end{itemize}
    \end{itemize}
  \end{multicols}
\end{frame}

\begin{frame}{History}
  \begin{block}{History}
    \begin{itemize}
    \item Ondřej Čertík started the project in 2006.
    \item Development took off in 2007 when SymPy first participated in Google
      Summer of Code. We have participated in every Google Summer of Code since.
    \item In 2011, Aaron Meurer (who also joined from Google Summer of Code) took
      over as lead developer.
    \end{itemize}
  \end{block}
\end{frame}

\begin{frame}{Present}
  \begin{block}{Current Status}
    \begin{itemize}
    \item Over 450 contributors.
    \item Current code base has over 400,000 lines of code and documentation.
    \item We have crossed the point of ``sympy a toy'' to ``sympy a tool''
    \end{itemize}
  \end{block}
\end{frame}

\begin{frame}{Future}
  \begin{block}{GSoC (1/2)}
    These are our current GSoC projects. Expect to see these features by the end
    of the summer.
    \begin{itemize}
    \item \normalsize Group Theory, \small Gaurav Dhingra
    \item \normalsize Extending solveset, \small Kshitij Saraogi
    \item \normalsize Completing Solveset, \small Shekhar Prasad Rajak
    \item \normalsize Implementation of Holonomic Functions, \small Shubham Tibra
    \item \normalsize Implementation of Singularity Functions to solve Beam Bending problems, \small Sampad Kumar Saha

    \end{itemize}
  \end{block}
\end{frame}

\begin{frame}{Future}
  \begin{block}{GSoC (2/2)}
    These are our current GSoC projects. Expect to see these features by the end
    of the summer.
    \begin{itemize}
    \item \normalsize Adding to SymEngine's Polynomial functionality and interfacing it with FLINT \& Piranha \small Srajan Garg
    \item \normalsize Implementing Finite Fields and Set module in SymEngine \small Nishant Nikhil
    \end{itemize}
  \end{block}
\end{frame}

\begin{frame}{Future}
\begin{block}{Other Plans}
\begin{itemize}
\item New assumptions
\item Make things faster
\item SymEngine (\url{https://github.com/symengine})
\item Implement more algorithms, so we can compute more things (and also make
  them faster)
\item Replacing \texttt{solve} with \texttt{solveset}
\item Encourage people to use SymPy for many applications
\item \url{https://github.com/sympy/sympy/wiki/gsoc-2016-ideas} for full list of
  things we want done
\end{itemize}
\end{block}
\end{frame}


%% I was too lazy to redo these for now

%% \begin{frame}{Git Commits Plots}
%%   \begin{block}{Last Year}
%%     \includegraphics[width=4in]{commits1.pdf}
%%   \end{block}
%% \end{frame}
%%
%% \begin{frame}{Git Commit Plots}
%%   \begin{block}{Last Year}
%%     \begin{itemize}
%%     \item The dotted line is 50 commits.
%%     \item Rough measurement of each project's ``bus factor''
%%     \end{itemize}
%%   \end{block}
%% \end{frame}
%%
%% \begin{frame}{Git Commits Plots}
%%   \begin{block}{All Time}
%%     \includegraphics[width=4in]{commits-all.pdf}
%%   \end{block}
%% \end{frame}
%%
%% \begin{frame}{Git Commit Plots}
%%   \begin{block}{All Time}
%%     \begin{itemize}
%%     \item SymPy has more total contributors\footnote{some of the other projects are actually exaggerated,
%%         because they don't use \texttt{.mailmap}}
%%     \item SymPy has a very welcome and friendly community, which is open, and
%%       actively encourages contributions.
%%     \item The SymPy code base is very approachable to new contributors.
%%     \item To be fair, Google Code-In accounts for a lot of this\ldots
%%     \end{itemize}
%%   \end{block}
%% \end{frame}


\begin{frame}{Projects Using SymPy}
\begin{itemize}
\item
  \href{http://www.sagemath.org/}{\textbf{Sage}}: A CAS, visioned to be
  a viable free open source alternative to Magma, Maple, Mathematica and
  MATLAB\@. Sage includes many open source mathematical libraries, including
  SymPy.
\item
  \href{https://cloud.sagemath.com}{\textbf{SageMathCloud}}:
  SageMathCloud is a web-based cloud computing and course management
  platform for computational mathematics.
\item
  \href{http://mathpix.com/}{\textbf{Mathpix}}: An iOS App, that detects
  handwritten math as input, and uses SymPy Gamma to evaluate the math input
  and generate the relevant steps to solve the problem.
\item
  \href{http://www.pydy.org/}{\textbf{PyDy}}: Multibody Dynamics with
  Python.
\item
  \href{http://openrave.org/docs/0.8.2/openravepy/ikfast/}{\textbf{IKFast}}:
  IKFast is a robot kinematics compiler provided by
  \href{http://openrave.org/}{OpenRAVE}. It analytically solves robot inverse
  kinematics equations and generates optimized C++ files. It uses SymPy for
  its internal symbolic mathematics.
\end{itemize}
  \end{frame}

\begin{frame}{Projects Using SymPy}
\begin{itemize}
\item
  \href{http://octave.sourceforge.net/symbolic/}{\textbf{Octave Symbolic}}:
  The Octave-Forge Symbolic package adds symbolic calculation features
  to GNU Octave. These include common CAS tools such
  as algebraic operations, calculus, equation solving, Fourier and
  Laplace transforms, variable precision arithmetic, and other features.
\item
  \href{https://github.com/brombo/galgebra}{\textbf{galgebra}}:
  Geometric algebra (previously \texttt{sympy.galgebra}).
\item
  \href{https://github.com/jverzani/SymPy.jl}{\textbf{SymPy.jl}}:
  Provides a Julia interface to SymPy using PyCall.
\item
  \href{https://mathics.github.io/}{\textbf{Mathics}}: Mathics is a
  free, general-purpose online CAS featuring Mathematica compatible
  syntax and functions. It is backed by highly extensible Python code,
  relying on SymPy for most mathematical tasks.
\item
  \href{http://sfepy.org/}{\textbf{SfePy}}: Simple finite elements in
  Python.
\end{itemize}
\end{frame}

\begin{frame}{Projects Using SymPy}
\begin{itemize}
\item
  \href{http://quameon.sourceforge.net/}{\textbf{Quameon}}: Quantum
  Monte Carlo in Python.
\item
  \href{http://lcapy.elec.canterbury.ac.nz/}{\textbf{Lcapy}}:
  Experimental Python package for teaching linear circuit analysis.
\item
  \href{http://digitalcommons.calpoly.edu/cgi/viewcontent.cgi?article=1072\&context=physsp/}{\textbf{Quantum
  Programming in Python}}: Quantum 1D Simple Harmonic Oscillator and
  Quantum Mapping Gate.
\item
  \href{http://mech.fsv.cvut.cz/~stransky/software/latexexpr/doc/}{\textbf{LaTeX
  Expression project}}: Easy \LaTeX{} typesetting of algebraic expressions
  in symbolic form with automatic substitution and result computation.
\item
  \href{https://www.researchgate.net/publication/260585491_Symbolic_Statistics_with_SymPy/}{\textbf{Symbolic
  statistical modeling}}: Adding statistical operations to complex
  physical models.
\end{itemize}
\end{frame}

\input{authors.tex}

\begin{frame}{Here at SciPy}
  \begin{block}{Talks}
    \begin{itemize}
    \item \normalsize Jason Moore, \textit{Simulating Robot, Vehicle, Spacecraft, and Animal Motion with Python (Advanced)}
      (Tutorial). \\ \footnotesize Monday 1:30 PM - 5:30 PM - Room 103
    \item \normalsize Aaron Meurer, Anthony Scopatz \textit{SymPy Code Generation}. \\ \footnotesize Thursday 11:30
      PM - 12:00 PM - Room 204
    \item \normalsize Ondřej Čertík, Isuru Fernando, Thilina Rathnayake, Abhinav Agarwal \textit{SymEngine: A Fast Symbolic Manipulation Library}. \\ \footnotesize Friday 3:30
      - 4:00 - Room 204
    \end{itemize}
  \end{block}
\end{frame}

%\begin{frame}{Here at SciPy}
%  \begin{block}{Bof}
%    \begin{itemize}
%      \item SymPy BoF - Wednesday 5:30 PM - 6:30 PM - Rm 203
%    \end{itemize}
%  \end{block}
%  \begin{block}{Sprints}
%    Come sprint with us!
%    \begin{itemize}
%    \item Releasing SymPy 1.1
%    \item Assumptions
%    \item Whatever interests you
%    \item Lot's of tasks that are easy for new contributors
%    \item Friday and Saturday
%    \end{itemize}
%  \end{block}
%\end{frame}

\begin{frame}
\Huge Let's begin!
\end{frame}
\end{document}
